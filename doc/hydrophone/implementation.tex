% Do NOT change this "Section" title
% and do NOT add more "Section" level titles.
\section{Implementation}\label{sec:implementation}
The implementation has the following parts :

% You can use how many "subsections" and "subsubsections" you like.
\subsection{Hydrophone array}
The hydrophone that has been used in AUV is made by Aquarian Audio Products who has sensor that capable of picking up sounds from below 20Hz to over 100KHz. Based on the time of arrival of signals at each hydrophone we are able to locate the sound source. When multiple sound sources are present, beamforming technique can be used to locate the individual sound sources. \newline
With an array of 4 hydrophones we are able to localize the sound source. The arrangement chosen is a rectangular array of hydrophones. The length and width of the rectangle is 20mm and 15mm respectively. Since the hydrophone array should detect sound frequencies of upto 30kHz, and considering the speed of sound in water to be 1500 m/s we can calculate the minimum distance between the hydrophones to be 
\begin{center}
$\dfrac{1500}{30000 * 2}$ = 25mm.
\end{center}

The advantages of chosing a rectangular array are:\begin{enumerate}
\item The multilateration algorithm is simplified.
\item Easy to position the hydrophones on the AUV.
\end{enumerate}
The disadvantages of the rectangular arrangement is that the accuracy of the algorithm decreases as the source comes closer to the plane of the hydrophones. 


\subsection{Hydrophone Circuit}
The hardware section include three circuits :
\begin{itemize}
   \item { \em Bridge amplifier circuit.}
   \item { \em Low pass filter. }
   \item { \em Schmitt trigger circuit. }
\end{itemize}
\subsubsection{Amplifier circuit }
The hydrophone cct is powered by the +12V , -12V feed from the Generic CAN controller. Amplifier circuit used OP amp named OP42GPZ - OP AMP, this OP amplifier is fast precision JFET (( The junction gate field-effect transistor )) input opertianal amplifier.
According to typical performance characteristics and electrical characteristics, the gain bandwidth of this product has 10 Mhz, this band width is suitable to our hydrophone circuit, also OP amplifier is supplied with current less than 6 mA.
In the rectifier circuit design it taken a great care to attain high precision analog signal by using   schottky diode whose characterized by low voltage drop and fast switching time.
The result that has been achieved by using schottky diode is more accurate than using other type of diodes.  

\subsubsection{Low pass filter}
Since low pass filter allow just low frequencies to pass or attenuates frequencies higher than the cutoff frequency.
Hydrophone circuit will pass just limited range of frequencies to avoid noises or other sounds, any noise on the hydrophone circuit will decrease accuracy.    
 
 \subsubsection{Schmitt trigger  circuit}
The schmitt trigger circuit used in the circuit is the TL084BCNG4 - IC, OP AMP, this OP amplifier designed with higher slew rates 13V/µs, supply voltage +12v and 3 Mhz band width that cover the frequancy range with hydrophone circuit. 
The comparator that will achieved from Schmitt trigger will let << INSERT SOME TEXT >> 
 
%\subsubsection{Subsubsection1}
%Test of one column figure. It should be shown as close as possible to this
%text. If you can't see the figure its number is \ref{fig:one_column_figure}
%and located on page \pageref{fig:one_column_figure}.

%\begin{figure}[h]
%    \includegraphics[width=0.5\textwidth]{./figure/figureA.png}
%   \caption{Figure A}
%   \label{fig:one_column_figure}
%\end{figure}

\subsection{Hydrophone software}
The hydrophone software should take input the pulses generated by the hydrophone circuit and output the location of the sound source in the CAN bus. The hydrophone software can be divided into 3 modules:
\begin{enumerate}
\item \textbf{External Interrupt:} The hydrophone circuit generates pulses when it gets a sound signal that is above the threshhold. The hydrophone circuit is conencted to the external interrupt pins of the CAN controller. The external interrupt is configured to generate an interrupt on the rising edge. Whenever the interrupt occurs the time of the occurance of interrupt and the number of hydrophone is pushed in a queue. Whenever the read data is called the data is removed from the queue and returned in FIFO order.
\item \textbf{Multilateration Algorithm:} The multilateration algorithm is initailized by passing the velocity of sound and the length and width of the rectangular array of hydrophones. After initialization the algorithm is executed by calling the run function. The run function takes input the time of arrival of the signals at the hydrophones. Care must be taken that the order of the time of arrival is correct. For example the Ti is the time for hydrophone at (0,0) while Tl is the time for hydrophone at (l,b). The algorithm returns the sound source location in x,y and z co-orfinates in millimeters. Also a boolean variable is returned to tell if the position is correct or not.
\item \textbf{The main program:} The main program initializes the CAN and the external interrupts. It then reads data from the queue of the external interrupts. When it has the data for all the hydrophones it calls the run of the multilateration algorithm. If the valid position is returned it is put out on the CAN bus.
\end{enumerate}

\subsection{Multilateration algorithm simulation}
Since the clocks in the microcontroller have a maximum frequency so to analyze how the clock resolution effects the algorithm a simulation was done in MATLAB. The simulation takes input the length and width of the rectangular array of hydrophones, the velocity of sound and the location of sound source in 3 dimensions with one of the hydrophone as orign. Then it calculates the time it will take for the sound from the source to reach the hydrophones. The time is then rounded off to the accuracy desired (currently it is set to micro seconds). Then using the rounded off times the location of sound source is re-canculated. The errors in the original position and calcualted position is observed.\newline
From the simulations we observed that with AT90CAN128 which has a external clock of 16MHz is too slow to get a fairly accurate estimation of position. We observed that the minimum clock required to have acceptable errors in localization is 1 GHz. Hence the code was not tested on AT90CAN128.
%\subsubsection{Subsubsection1}
%