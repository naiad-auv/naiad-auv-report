% Project Management

% THIS IS SIGPROC-SP.TEX - VERSION 3.1
% WORKS WITH V3.2SP OF ACM_PROC_ARTICLE-SP.CLS
% APRIL 2009
%
% It is an example file showing how to use the 'acm_proc_article-sp.cls' V3.2SP
% LaTeX2e document class file for Conference Proceedings submissions.
% ----------------------------------------------------------------------------------------------------------------
% This .tex file (and associated .cls V3.2SP) *DOES NOT* produce:
%       1) The Permission Statement
%       2) The Conference (location) Info information
%       3) The Copyright Line with ACM data
%       4) Page numbering
% ---------------------------------------------------------------------------------------------------------------
% It is an example which *does* use the .bib file (from which the .bbl file
% is produced).
% REMEMBER HOWEVER: After having produced the .bbl file,
% and prior to final submission,
% you need to 'insert'  your .bbl file into your source .tex file so as to provide
% ONE 'self-contained' source file.
%
% Questions regarding SIGS should be sent to
% Adrienne Griscti ---> griscti@acm.org
%
% Questions/suggestions regarding the guidelines, .tex and .cls files, etc. to
% Gerald Murray ---> murray@hq.acm.org
%
% For tracking purposes - this is V3.1SP - APRIL 2009

\documentclass{acm_proc_article-sp}

% For proper urls in the reference list
\usepackage[hyphens]{url}

\begin{document}

\title{Project Management}
\subtitle{How to alienate friends}
%\titlenote{A full version of this paper is available as
%\textit{Author's Guide to Preparing ACM SIG Proceedings Using
%\LaTeX$2_\epsilon$\ and BibTeX} at
%\texttt{www.acm.org/eaddress.htm}}}
%
% You need the command \numberofauthors to handle the 'placement
% and alignment' of the authors beneath the title.
%
% For aesthetic reasons, we recommend 'three authors at a time'
% i.e. three 'name/affiliation blocks' be placed beneath the title.
%
% NOTE: You are NOT restricted in how many 'rows' of
% "name/affiliations" may appear. We just ask that you restrict
% the number of 'columns' to three.
%
% Because of the available 'opening page real-estate'
% we ask you to refrain from putting more than six authors
% (two rows with three columns) beneath the article title.
% More than six makes the first-page appear very cluttered indeed.
%
% Use the \alignauthor commands to handle the names
% and affiliations for an 'aesthetic maximum' of six authors.
% Add names, affiliations, addresses for
% the seventh etc. author(s) as the argument for the
% \additionalauthors command.
% These 'additional authors' will be output/set for you
% without further effort on your part as the last section in
% the body of your article BEFORE References or any Appendices.

\numberofauthors{1}
% I've updated the number of authers ~ Christoffer 2012-10-14

%  in this sample file, there are a *total*
% of EIGHT authors. SIX appear on the 'first-page' (for formatting
% reasons) and the remaining two appear in the \additionalauthors section.
%
\author{
% You can go ahead and credit any number of authors here,
% e.g. one 'row of three' or two rows (consisting of one row of three
% and a second row of one, two or three).
%
% The command \alignauthor (no curly braces needed) should
% precede each author name, affiliation/snail-mail address and
% e-mail address. Additionally, tag each line of
% affiliation/address with \affaddr, and tag the
% e-mail address with \email.
%
% 1st. author
\alignauthor
Simon Elgland\\
       \email{simon.elgland@gmail.com}
}
% There's nothing stopping you putting the seventh, eighth, etc.
% author on the opening page (as the 'third row') but we ask,
% for aesthetic reasons that you place these 'additional authors'
% in the \additional authors block, viz.
% \additionalauthors{Additional authors: John Smith (The Th{\o}rv{\"a}ld Group,
%email: {\texttt{jsmith@affiliation.org}}) and Julius P.~Kumquat
%(The Kumquat Consortium, email: {\texttt{jpkumquat@consortium.net}}).}
%\date{30 July 1999}
% Just remember to make sure that the TOTAL number of authors
% is the number that will appear on the first page PLUS the
% number that will appear in the \additionalauthors section.

\maketitle
%\begin{abstract}
   The Sensor Fusion of the Naiad AUV is done on a BeagleBone Black which takes care of
   computationally heavy calculations. Data is received from
   both an ethernet connection and a CAN bus. The data is received as
   CAN messages independently of underlying technology in the current design.
   For long term maintainability of the code the Sensor Fusion code is divided
   into seven Ada tasks where each task has a specific responsibility.
\end{abstract}

% A category with the (minimum) three required fields
% \category{A.1}{General Literature}{Introductory and Survery}
%A category including the fourth, optional field follows...
% If we want to add another category (or several).
%\category{D.2.8}{CHANGE THIS Software Engineering}{Metrics}[complexity measures, performance measures]

% \terms{Theory}

% \keywords{Threat Models, Centralised systems, Decentralised systems, Censorship, Privacy, Natural disasters} % NOT required for Proceedings

% Do NOT change this "Section" title
% and do NOT add more "Section" level titles.
\section{Introduction}\label{sec:introduction}
The primary objective of the vision system is to detect target objects of various shapes and colours and communicate to the other components within the robot to take necessary actions. The system comprises of two GIMME-2 stereovision boards \cite{gimme2manual} equipped with a hard processor \cite{hardp} called Zynq 7020 from Xilinx and two 10 Megapixel OV10810 \cite{web:OV10810} image sensors. The Zynq processor consists of an ARM Cortex A9 dual-core processor and a FPGA in the same silicon chip. The FPGA architecture augments the image processing capabilities of the system as it can do arithmetic operations in parallel, unlike the CPUs which do it sequentially. The board has 8 GB of RAM for the Processing System (ARM processor) and 4 GB of RAM for Programmable Logic (FPGA). There are two fast Ethernet ports and one gigabit Ethernet port for high speed communication with other devices. There is a slot for external memory storage where the image processing binary files are stored. The operating system installed on the GIMME-2 board is Linux (version 3.6.0-xilinx). 

%\input{method.tex}
%% Do NOT change this "Section" title
% and do NOT add more "Section" level titles.
\pagebreak
\section{Implementation}\label{sec:implementation}
As mentioned in Section~\ref{sec:introduction}, the Sensor Fusion is divided into two parts: Software architecture and Sensor fusion calculations. Sensor fusion calculations are in turn divided into attitude
calculations and dead reckoning.

\subsection{Software architecture}
To meet the software requirements it was decided to use Ada Tasks and Ada
Protected Objects. Protected objects are a special type of objects within Ada
to facilitate shared memory in a type safe way. A requirement from the
customer was the use of the Ada Ravenscar profile. This profile limits
the use of tasks to make the program easier to verify and validate. This meant
that each protected object in use was only allowed to be called from no more than two
different tasks.

The common data structure within the Naiad AUV is the CAN message structure,
it has a message ID and payload. These messages were modelled as objects within
the Sensor fusion code and the CAN message type is the only data type that
is passed around between Ada Tasks.

To manage the incoming and outgoing data two tasks were set up for each link,
TCP\_IN and TCP\_OUT for the ethernet connection as well as CAN\_IN and CAN\_OUT for
the CAN bus connection. Each one of these tasks had only one responsibility, which
was to either send or receive CAN messages.
The next task to be added was the Main
task which would do all the calculations. The design in this stage had five
different tasks with distinct responsibility, though one main problem left to solve
was the requirement that in this design the Main task would have to do a lot of
filtering of CAN messages both to and from the TCP and CAN connection. To solve this
two new tasks were introduced, the TCP\_IN\_FILTER and CAN\_IN\_FILTER. The filtering
tasks would be given a set of CAN message IDs on boot up that were of interest to
the Main task. The Main task had now only one responsibility left and that was
to do the actual sensor fusion calculations.

\begin{figure}[ht]
    \includegraphics[width=0.5\textwidth]{./figure/software_architecture.png}
    \caption{Sensor fusion software architecture. Showing the different tasks and
    how they interact with each other through several different protected objects.}
    \label{fig:software_architecture}
\end{figure}

The final design of the software architecture is seen in Figure \ref{fig:software_architecture}.
The dashed circles on the left and right side are protected objects specific for
the different hardware resources. For the CAN bus this is required because one
cannot read and write at the same time, but for the TCP connection multiple connections
can be done in parallel so this can be changed.


% You can use how many "subsections" and "subsubsections" you like.
\subsection{Sensor fusion calculations}
The Sensor fusion calculations consist of two parts, attitude calculations and
dead reckoning.

\subsection{Attitude Calculations}
The Vectornav VN100 IMU (Inertial Measurement Unit)~\cite{web:vn100} used in 
the Naiad AUV gets the yaw, pitch and roll values by integrating over the
angular velocities. However even when the IMU is kept still it still has
some residual angular velocities which causes the yaw, pitch and roll to drift
over time and hence need to be corrected.
The IMU corrects its roll and pitch values using its accelerometer 
readings which give the direction of gravity.
Once roll and pitch are corrected, the magnetometer reading is used to correct
the yaw value.

The Vectornav VN100 IMU has a Kalman
filter implemented which provides us with the compensated yaw, pitch and roll
values. However according to the requirements the magnetic fields cannot be
trusted in some competitions. This leads to the yaw values being unreliable.
Thus to correct yaw, a Fiber Optic Gyroscope (FOG)~\cite{web:fog1} is used.

A fiber optic gyroscope uses a laser interferometer to calculate the angular
speed of the gyroscope. The challenge is to
integrate the readings obtained from the FOG to the readings obtained from
the IMU. If one integrates the angular velocity from FOG one can get the yaw value, 
this value tells the angle the AUV rotates in the z axis attached to the AUV.
The idea is that this is the yaw angle that should be applied to the AUV after
the roll and pitch is done, that is the yaw should be the last operation to
reach from one state to other.

The IMU can output the yaw, pitch and roll in body frame of reference.
To use the FOG value it should be converted to roll, pitch and yaw order.
After that the yaw value of IMU can be replaced with the yaw from the FOG.
The following two methods were proposed to change the angle order from yaw,
pitch and roll to roll, pitch and yaw:
\begin{enumerate}
\item The idea is that for small changes in yaw, pitch and roll angles,
    the order in which the rotations are applied does not matter upto first
    degree of accuracy. So changes to yaw pitch and roll are calculated for
    IMU and the yaw change for FOG at each time step. If it is assumed that
    the AUV moves slowly and that the samples are taken fast enough, the changes
    to the angles will be small. Thus the rotation matrix from the previous
    step is multiplied with the rotation matrix obtained by combining the yaw
    from FOG and roll and pitch from IMU to get a new rotation matrix.
\item VN100 IMU can also output the Directional Cosine Matrix (DCM). The other method is to
    get the DCM and compare it to the rotation matrix obtained by
    applying rotations in roll, pitch and yaw order. Then one can derive
    formulas for the roll, pitch and yaw in terms of DCM matrix elements.
\end{enumerate}

\subsection{Dead Reckoning}
Dead reckoning is the name of the method used to calculate the x, y and z
coordinates of the AUV with the starting position as the origin.
The simplest way to do this is to integrate the accelerometer values
twice to get the position. However this can lead to large amounts of errors
because a small error in accelerometer increases exponentially with
time as it is multiplied with time squared. To avoid this the plan was
to use front and down facing cameras, which have stereo vision, and
calculate the velocity of the AUV. Then the velocity and acceleration
can be combined using Kalman filter to get a better estimate of the position.

%
\section{Result}\label{sec:result}
Since the project ran out of time not all of the goals set for the Sensor Controller were achieved. 

\subsection{Temperature sensor}
A small PCB prototype which contains the converter circuit was built and connected to the temperature sensor and the Generic CAN controller. The tests results were very good. The entire temperature range was not tested, only from +15 degrees Celsius to +100 degrees Celsius due to the lack of equipment to create the rest of the temperature range, but for this range the sensor performed very well. 

\subsection{Pressure sensor}
The pressure sensor was never put through any major tests. This was due to time limits and the inherent difficulty of creating sufficient levels of pressure to perform a test. \\
The only practical way to do this is to submerge the sensor in water. The depth needed\footnote{Approximately 20 meters to get an absolute pressure of 3 bars.} to test the whole range of pressure would mean that either a very long cable would need to be attached to the sensor (and made waterproof), or, that one puts the sensor, the Sensor controller and some hardware for  logging the sensor readings in a waterproof hull. The latter could be easily be done once the AUV's hull is completed, but this stage was not reached during the duration of the project.

A simple test was done where it was connected to a voltage source and its output voltage measured. The result from this test was good. However, more testing and calibration will be needed.

\subsection{Salinity sensor}
Since the salinity sensor was never purchased, no results regarding it were obtained.


\input{learnings.tex}
%\section{Conclusion}\label{sec:conclusion}
Project courses such as this project course are good since they let each student use his/her skills in a realistic context, cooperate with others and do something that has not been done before. Because of this it is important that each student has a solid base of skills in his/her area(s) before starting such a project and that cooperation within the project group works well.
\pagebreak

As for my advice for a plan for future projects I recommend to use the first two weeks for analyzing the requirements given by the customer and for State-of-the-Art research. Followed by two weeks for defining the structure of the robot, its subsystems and the interfaces between them. Throughout this process one should keep a close contact with the customer (the person(s) or entity that orders the robot) in order to make sure that the structure decided upon is what the customer wants. \\
Once all subsystems are defined, at least one person should be assigned as responsible for each subsystem (one person can be responsible for several subsystems).

The next phase of the project would then be to define the structure of each subsystem in order to get an initial understanding for how each subsystem should be implemented. When this has been done it might very well become apparent that one or several subsystems can not be implemented in the intended way and/or that the interface of a subsystem has to be changed.\emph{ If this happens, then this would be something good.} The fault in the overall structure of the robot can then be detected at an early state before any major implementations have been done and consequently lots of time can be saved this way. \\
After this phase comes the actual implementation phase. Testing should be done continuously so that any errors or flaws are detected as early as possible. When several subsystems that are to work together have been implemented, one should try to test them together to verify that their common interface works (that they work together). As more and more subsystems are finished, more and more complex (and accurate) tests can be done.

Lastly, remember to document all work in a proper way. Your work is of little use if nobody else can understand or use what you have built during the project.

%\end{document}  % This is where a 'short' article might terminate

% Just comment this out if we don't need it.
%\input{acknowledgment.tex}

%
% The following two commands are all you need in the
% initial runs of your .tex file to
% produce the bibliography for the citations in your paper.
%\bibliographystyle{abbrv}
%\bibliography{../references}  % references.bib is the file with all references. 
% You must have a proper ".bib" file
%  and remember to run:
% latex bibtex latex latex
% to resolve all references
%
% ACM needs 'a single self-contained file'!
%
%APPENDICES are optional
\balancecolumns
%\appendix
%% Each section is a new appendix
\section{Appendix}
\subsection{Getting started with GIMME-2 board}\label{sec:first_appendix}
This section describes the step by step procedure to get started with the GIMME-2 board. The operating system used on the host machine is Ubuntu 12.04. The prerequisites for initializing the GIMME-2 board are:\begin{enumerate}
\item Micro-USB Male (Type B) to USB Male (Type A) connector or an Ethernet cable.
\item Power supply. The board should be supplied with voltages in the range +12V to +24V.
\item If the user is planning to use the USB cable, then a serial terminal, for example \textit{gtkterm}, should be installed on the host PC to interact with the Linux on the GIMME-2 board.
\end{enumerate}

Assuming that Zynq Linux is already installed on the GIMME-2 board, the user can interact with it in either of the two ways described below. Ensure that the jumper configuration is 0100 before powering on the board.
\subsubsection{Using serial interface}
Connect the micro-USB to USB connector to the port labelled \textit{USB0} on the board and connect the other end to the host PC. Set the voltage of the power supply to +12V and connect it to the GIMME-2 board. Now open the serial terminal and the booting steps can be seen there. Once the booting is over, the user gets a \texttt{zynq>} prompt on the screen.

\subsubsection{Using Ethernet}
Connect one end of the Ethernet cable to the middle port among the three on the GIMME-2 board and the other end to the host machine. Now open the bash terminal and execute
\texttt{ssh root@192.168.1.10}. Type in \textit{root} when password is prompted. The user gets a \texttt{zynq>} prompt on the screen.

\subsection{Creation of Zynq Boot Image}
The procedure for creating a boot image is quite complex and time consuming. For the sake of the new users, all the files required to create a boot image are available on the Subversion repository for project Naiad \cite{web:svnGimme}. Additional to the prerequisites mentioned in section 5.1, the user needs to do the following:\begin{enumerate}
\item Install the full package of Xilinx ISE Design Suite\cite{web:downloadISE} including the Software Development Kit (SDK) on the host PC.
\item Install the driver for Xilinx Platform USB Cable II \cite{web:driverCable}. If Step 1 is done in Windows OS, then no need to install the driver as it comes with the package.
\end{enumerate}

Using SDK, new boot image can be created by giving offsets to the binary files as mentioned in the user guide \cite{web:svnGimme}. If amendments have to be made to the Linux kernel like installing a new Linux driver, then the RAM disk image should be modified and a new boot image should be created with the new RAM image. All the steps for modifying RAM image are also detailed in the user guide \cite{svnGimme}.

%\section{Headings in Appendices}
%The rules about hierarchical headings discussed above for
%the body of the article are different in the appendices.
%In the \textbf{appendix} environment, the command
%\textbf{section} is used to
%indicate the start of each Appendix, with alphabetic order
%designation (i.e. the first is A, the second B, etc.) and
%%a title (if you include one).  So, if you need
%hierarchical structure
%\textit{within} an Appendix, start with \textbf{subsection} as the
%highest level. Here is an outline of the body of this
%document in Appendix-appropriate form:
%\subsection{Introduction}
%\subsection{The Body of the Paper}
%\subsubsection{Type Changes and  Special Characters}
%\subsubsection{Math Equations}
%\paragraph{Inline (In-text) Equations}
%\paragraph{Display Equations}
%\subsubsection{Citations}
%\subsubsection{Tables}
%\subsubsection{Figures}
%\subsubsection{Theorem-like Constructs}
%\subsubsection*{A Caveat for the \TeX\ Expert}
%\subsection{Conclusions}
%\subsection{Acknowledgments}
%\subsection{Additional Authors}
%This section is inserted by \LaTeX; you do not insert it.
%You just add the names and information in the
%\texttt{{\char'134}additionalauthors} command at the start
%of the document.
%\subsection{References}
%Generated by bibtex from your ~.bib file.  Run latex,
%then bibtex, then latex twice (to resolve references)
%to create the ~.bbl file.  Insert that ~.bbl file into
%the .tex source file and comment out
%the command \texttt{{\char'134}thebibliography}.
% This next section command marks the start of
% Appendix B, and does not continue the present hierarchy
%\section{More Help for the Hardy}
%The acm\_proc\_article-sp document class file itself is chock-full of succinct
%and helpful comments.  If you consider yourself a moderately
%experienced to expert user of \LaTeX, you may find reading
%it useful but please remember not to change it.
%\balancecolumns
% That's all folks!
\end{document}
