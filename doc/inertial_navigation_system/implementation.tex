
\section{Implementation}\label{sec:implementation}
The sensors used were the VN-100 inertial measurement unit from VectorNav Technologies and the 8088 000-112 Fiber optic gyro from SAAB. The VN-100 (refereed to as the "IMU") is a digital sensor and provides a serial interface. The Fiber optic gyro (refereed to as the "FOG") is an analog sensor and gives a analog voltage output proportional to the angular speed of rotation. 

The software running on the Generic CAN controller is called AT90CAN\_Ins\_Controller and handles communication with the IMU and the ADC that is connected to the FOG as well as outputs the measurements from the sensors onto the CAN bus.

\subsection{Inertial measurement unit}
The IMU is powered by the +5V feed from the Generic CAN controller. It provides a serial interface at both 3.3 Volt TTL levels and RS-232 levels, neither of which can be directly connected to Generic CAN controller since UART busses on the AT90CAN128 microcontroller works at +5 Volt TTL levels. A level change was consequently needed. The MAX232 converter was chosen to convert from RS-232 levels to +5 Volt TTL levels since it needs only a 5 Volt supply. \newline
The IMU has one output pin called \emph{SYNC\_OUT} that can be configured to output a synchronization pulse whenever the IMU has completed a measurement. The SYNC\_OUT pin is connected to Port, D pin 2 on the AT90CAN128. This pin is activates the \emph{external interrupt 2}. \newline
The above configuration enables an interrupt service routine to be activated every time the IMU has made a measurement.\newline 
\emph{Please note that this pin also is the receive pin on the UART1 bus.} 

\subsubsection{Interface}
As mentioned, the IMU communicates over a serial interface and is connected to the UART0 bus of the AT90CAN128. 


\subsection{Fiber optic gyroscope}
Since the Fiber optic gyroscope is a high precision analog sensor great care has to be taken when designing its interface circuit. \newline
The accuracy of the FOG is highly dependent on the power supply, any noise on the power supply will decrease accuracy. The same need for accuracy is present in the analog to digital conversion. Fortunately, the manufacturer provided the Naiad AUV project with a schematic for the circuit that is to be used for the analog to digital conversion. The analog to digital converter (ADC) used was the ADS1255 extremely low-noise, 24-bit analog to digital converter which has an SPI interface.

\subsubsection{Power supply}
The FOG has three power lines, +12 Volts, -12 Volts and +5 Volts as well as a power ground and a signal ground. \newline
According to the manual of the FOG, the +5 Volt line is not allowed to be present without the +12 Volts and -12 Volts lines. This is ensured by using a PVG612 Photovoltaic Relay that is activated by a digital output from the AT90CAN128. This pin is set high after a delay after start up. \newline
In order to stabilize the supply voltage several capacitors are connected between each supply line and ground. These capacitors are placed as close to the FOG connector as possible and have very low equivalent series resistance (ESR) in order to remove as much noise as possible. 

Although the Generic CAN controller could supply the INS board with +12 Volts, this supply is not galvanically isolated from the supply voltage. For this reason only the 5 Volt supply, which is galvanically isolated, was used by the INS controller and consequently DC to DC conversion from 5 Volts to +12 and -12 Volts was needed. \newline
The User manual of the FOG specifies that low noise switching DC to DC converters with a switching frequency of at least 400 kHz is to be used. The LT1931 inverting DC to DC converter was used to supply -12 Volts. The LMR62014 step-up voltage regulator was chosen to supply +12 Volts. Both of which use switching frequencies well over 1 MHz.


\subsubsection{Analog to digital conversion}
Before analog to digital conversion, the analog output from the FOG is filtered to remove high frequency components that could otherwise give rise to aliasing distortion. This filter is included in the schematic provided by the manufacturer of the FOG and was not modified during the project. \newline
The THS4131 differential amplifier is the central component of the anti-aliasing filter. 

The analog to digital converter used in the circuit is the ADS1255. The ADS1255 is run at <INSERT SAMPLE RATE> samples pers second and has a resolution of 24 bits, however only about <INSERT EFFECTIVE RESOLUTION> bits is at this sample rate. \newline
The ADS1255 is supplied by both 5 Volts and as well as 3.3 Volts. For this reason the LM1117MP-3.3 linear voltage regulator was used to convert from 5 Volts to 3.3 Volts. 

Both the ADS1255 and the THS4131 require a voltage reference of 2.5 Volts. This voltage level needs to be very stable and noise free or accuracy of the whole circuit could be affected. The ADR421ARMZ ultraprecision, low noise voltage reference was used for this task.

\subsubsection{Interface}
The analog to digital converter communicates with the AT90CAN128 using SPI...  <INSERT MORE TEXT ABOUT THE SPI COMMUNICATION WITH THE ADC!!!>







\newpage