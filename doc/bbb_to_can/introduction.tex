
\section{Introduction}\label{sec:introduction}
All higher level computations run on BeagleBone Black boards (abbreviated "BBB" in this report), therefore these boards must be connected to the CAN-bus of Naiad. Research in the beginning of the project found that there was a \emph{Cape} (an extension board to the BBB) available, that could connect a BBB to the CAN bus. However, further investigation revealed that this cape was facing issues and may not work. For this reason, it was decided that each BeagleBone Black would be connected to the CAN bus using a board custom built during the project, the Generic CAN controller. \newline

The Generic CAN controller is a small (approx. 76 by 40 millimetres) electronic board that has an AT90CAN128 microcontroller~\cite{web:at90can}, an MCP2551 CAN transceiver~\cite{web:mcp2551} and all the peripheral circuitry needed for these, as well as its own power supply. The Generic CAN controller is used throughout the project for several tasks. It can be connected to the CAN bus and has two UART buses as well as an SPI bus. \newline
Hence there were two possible ways to connect the BBB to the Generic CAN controller: via UART or SPI. \newline
UART was chosen for two reasons: In the Vasa project, a protocol for sending CAN messages over UART was already implemented for an AT90CAN128 microcontroller (the same that is used on the the Generic CAN controller) which was used as a starting point. Furthermore, the UART allows the communication in one direction to be truly independent of that in the other direction. Using SPI would most definitely be possible but would require the communication protocol between the BBB and the Generic CAN controller to be more complicated and would have to be implemented from scratch.  \newline
The protocol used from the Vasa project was modified to better fit the requirements of the Naiad project.

Whenever the BBB needs to send a CAN message, the message is converted into a string of bytes that will be sent over UART to the Generic CAN controller that in turn will convert this string of bytes back to a CAN message and output it on the CAN bus. When the Generic CAN controller receives a CAN message the same process will be done but in the opposite direction. 
