% Do NOT change this "Section" title
% and do NOT add more "Section" level titles.
\section{Implementation}\label{sec:implementation}
The latest solution presented for IP over CAN is from Ditze et. al. with a 10
bits prioritisation field, 3 bits message type field, 8 bits destination
address and 8 bits sender address. It's from this solution work started in the
Naiad project.

As the main goal for Naiad project was to use Space plug-and-play Avionics
technology a requirement was that address assignment had to be done
dynamically. A first suggestion was to use a common message ID that all new
nodes transmit a DHCP request to with the requesting host's identifier in the
payload. The problem with this approach is when multiple hosts boots up
simultaneously, collision will occur and the CAN Bus will go into an error
state.

Next approach was to use the destination address field and sender address field
with a total of 16 bits from the message ID as a unique host identifier field
when sending a DHCP request message. This approach seems to be viable but 16
bits will only give a few more than 2000 addresses which might be too few
to prevent accidental CAN Bus collisions during boot up.

At this point other priorities were made in the project so no further progress
was done.
