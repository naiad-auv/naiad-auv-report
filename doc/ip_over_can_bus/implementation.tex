% Do NOT change this "Section" title
% and do NOT add more "Section" level titles.
\section{Implementation}\label{sec:implementation}
The latest solution presented for IP over CAN is from Ditze et. al. with a 10
bits prioritisation field, 3 bits message type field, 8 bits destination
address and 8 bits sender address. It's from this solution work started, in the
Naiad project.

<insert image of 29 bits message id mapped to the IP over CAN suggestion>

In their solution they suggest a gateway that routes traffic to and from the
CAN Bus network. The routing that had to be done would map full 32 bits IP
addresses to CAN Bus IP addresses of 8 bits. This would be done by dropping the
first 24 bits of the IPv4 address and only using the last 8 bits over the CAN
Bus.

Ditze et. al. doesn't go into
detail about assignemnt of IP addresses
but instead mentions that it can be done statically or dynamically. As the main
goal for Naiad project was to use Space plug-and-play Avionics
technology a requirement was that address assignment had to be done
dynamically. A first suggestion was to use a well-known message ID that all new
nodes transmit a DHCP request with. As the payload for the DHCP request a
host's identifier would be set.
The problem with this approach is when multiple hosts boot up
simultaneously, collision will occur and the CAN Bus will go into an error
state.

Next approach was to use the destination address field and sender address field
with a total of 16 bits from the message ID as a unique host identifier field
when sending a DHCP request message. This approach seems to be viable and would
give 16 bits addresses which might be too few to prevent accidental CAN Bus
collisions during boot up. The 16 bit identifier would have to be generated
from some kind of hardware ID such as the MAC Address for ethernet devices.

<insert image of 29 message id bits mapped to destination/sender adress and
another step mapping it to host identifier field during DHCP request>

At this point other priorities were made in the project so no further progress
was made.
