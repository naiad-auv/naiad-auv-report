% Do NOT change this "Section" title
% and do NOT add more "Section" level titles.
\section{Introduction}\label{sec:introduction}
As a part of the Space plug-and-play Avionics requirement one part turned out
to focus on IP over CAN.

% You can use how many "subsections" and "subsubsections" you like.
\subsection{Internet Protocol (IP)}
The Internet Protocol is the main protocol for addressing different hosts on a
single or across multipe link networks. Internet Protocol version 4 is the
version currently most users use but with no more addresses available
for new devices a move to Internet Protocol version 6 is currently ongoing.

\subsubsection{IP version 4 (IPv4)}
IPv4 originally defined as a standard in September 1981 \cite{web:rfc791} has
seen a few updates throughout the years. The core part is that each host
connected to a network has a 32 bit address. For human readability this address
is often written in decimal form as four octets (e.g. 192.168.0.1).

As the number of connected devices grew more than expected in the nineteen-nineties
"Network Address Translation" (NAT) was introduced to limit the number of
publically accessible IP addresses assigned and improve routing for Internet
Service Providers \cite{web:rfc1631, web:rfc1918, web:rfc3022}.

One of the most common ways to assign IP addresses on a network is through the
Dynamic Host Configuration Protocol (DHCP) \cite{web:rfc2131, web:rfc2132, web:rfc4361}.

\subsubsection{IP version 6 (IPv6)}
text
\subsection{Controller Area Network (CAN)}
text
