% Do NOT change this "Section" title
% and do NOT add more "Section" level titles.
\section{Introduction}\label{sec:introduction}
As part of the Space plug-and-play Avionics requirement
focused on IP over CAN. The following sections will give a short
introduction to the Internet Protocol (IP) and the Controller Area Network
(CAN or CAN bus).

% You can use how many "subsections" and "subsubsections" you like.
\subsection{Internet Protocol (IP)}
The Internet Protocol is the main protocol for addressing hosts on a
single, or multipe link networks. Internet Protocol version 4 is the
version currently most hosts use but with no more addresses available
for new devices a move to Internet Protocol version 6 is currently ongoing
\cite{web:rfc6540}.

\subsubsection{IP version 4 (IPv4)}\label{introduction:ipv4}
IPv4, originally defined as a standard in September 1981 \cite{web:rfc791}, has
seen a few updates throughout the years. The core part is that each host
connected to a network has a 32 bit address. For readability this address
is often written in decimal form as four octets (e.g. 192.168.0.1).

As the number of connected devices grew more than expected in the 1990's
"Network Address Translation" (NAT) was introduced to limit the number of
publically accessible IP addresses assigned and at the same time improve routing for Internet
Service Providers \cite{web:rfc1631, web:rfc1918, web:rfc3022}.

One of the most common ways to assign IP addresses on a network is through the
Dynamic Host Configuration Protocol (DHCP) \cite{web:rfc2131, web:rfc2132, web:rfc4361}.
When a host connects to the network, the client sends a DHCP request and the DHCP
server(s) available on the network replies back with a DHCP offer containing a free
IP address the new host can use. As identification the client uses either its
MAC Address or another client's identification value depending one the version of the
protocol that is used. After this process end-to-end communication is possible
within the network.

\subsubsection{IP version 6 (IPv6)}
IPv6 was first specified as proposed standard RFC 1883 \cite{web:rfc1883} in
December 1995 and now obsoleted by RFC 2460 \cite{web:rfc2460}. The motiviation for
its creation is that it was in the end of 1980's apparent that the
IPv4 address space was too small. Instead of the 32 bits addresses in IPv4, an
IPv6 address consists of 128 bits. For readability this address is
written in hexadecimal form in 8 groups separated by colons
(e.g. 2a00:860:340:aabf:aabf:baad:3423:495).

RFC 4291 defines the IPv6
addressing architecture \cite{web:rfc4291} while RFC 5952 \cite{web:rfc5952}
defines recommended text representation of IPv6 addresses.

In comparison with IPv4 the address assignment can be done in more than two different
ways. The first way is by static assignemt which is equal to IPv4.
The second way is DHCPv6 that works in a similiar manner
to DHCP \cite{web:rfc3315}. As a configuration option of DHCPv6, a third
assignment procedure is possible with "DHCP
Prefix Delegation" (DHCP-PD) \cite{web:rfc3633} that assigns complete blocks of
addresses to routers which are then free to distribute addresses within that
block of addresses. If DHCPv6 isn't available "Stateless Address
Autoconfiguration" (SLAAC) is a way for each host to generate and set a unique
address for each of its interfaces.
This also includes what is called "Duplicate Address Detection" to prevent
multiple hosts from using the same IPv6 address \cite{web:rfc4862, web:rfc4941}.

As more and more devices get connected to the internet, the "Internet of things" become a
reality. For "Internet of things" most focus has been put on wireless
technologies such as the 6LoWPAN \cite{web:ietf_charter_6lowpan} work and
ROLL \cite{web:ietf_charter_roll} within the IETF.

In April 2012 the IETF released a Best Current Practice (BCP) document
"IPv6 Support Required for All IP-Capable Nodes". This BCP recommends that
IPv6 must be implemented in all hosts that are IP capable \cite{web:rfc6540}.

\subsection{Controller Area Network (CAN)}
The CAN protocol is a message based broadcast protocol developed and specified
by ROBERT BOSCH GmbH (Bosch). CAN Specification 2.0 was released in September
1991 and consists of two parts, A and B. An implementation of the CAN
Specification must comply with either of the two parts or both. The main
difference is the length of message IDs which in the standard format is 11 bits
long and in the extended format is 29 bits long. Part A requires support for
the standard format while Part B implementations must support
standard format and extended format \cite{standard:can_bus}.

The key part of the CAN bus protocol is that all hosts connected to the same
bus will be able to read simultaneously from the bus, it's up to each host to
filter in the wanted messages. The filtering process is done by specifying one
or multiple masks on each host that is used when reading incoming message
IDs.

One important aspect is that if multiple hosts transmit a message at the same
time with the same message ID but with different payloads the CAN bus will
enter an error state. Multiple hosts are therefore not allowed to transmit messages with the
same message ID.

CAN with Flexible Data-Rate (CAN FD) is a new specification which increase the
data rate from a maximum of 1Mbit/s to 8Mbit/s \cite{standard:can_bus_fd}.

\subsection{IP over CAN}
Previous work in this field is limited. In 2001 Petr Cach and Petr Fiedler
created a first draft for IP over CAN \cite{web:draft-ip_over_can}. They
clearly state that their solution is not for hard real-time operations. In
2003 Ditze et. al. \cite{web:porting_ip_can} took another approach with a larger prioritisation
band.

The following parts of the report is structured as follows. Section \ref{sec:implementation}
goes through what has been done with IP over CAN in the Naiad project and section
\ref{sec:conclusion} focuses on lessons learned and future work.
